\documentclass[10pt, letter]{article}

\usepackage[utf8]{inputenc}
\usepackage{csquotes}
\usepackage[english]{babel}
\usepackage{microtype}
\usepackage[T1]{fontenc}
\usepackage{lmodern}
\renewcommand*\familydefault{\sfdefault} %% Only if the base font of the document is to be sans serif

\usepackage{xcolor}
\definecolor{urlblue}{HTML}{01256e}
\usepackage[inline]{enumitem}
\setitemize{noitemsep,topsep=-.2cm,parsep=0pt,partopsep=0pt}

\usepackage{lipsum}% http://ctan.org/pkg/lipsum

\usepackage{booktabs}
\usepackage{array}
\usepackage{makecell}

% Spacing between paragraphs
\usepackage{parskip}
\newlength{\currentparskip}

%%%%% Indentation of content
\newlength{\cvindent}
\setlength{\cvindent}{1cm}

% bibliography setup
\usepackage[maxbibnames=99,sorting=none]{biblatex}
\addbibresource{ref.bib}
\defbibenvironment{bibliography}
  {\list
     {\printtext[labelnumberwidth]{%
      \printfield{labelprefix}%
      \printfield{labelnumber}}}
     {\setlength{\labelwidth}{\labelnumberwidth}%
      \setlength{\leftmargin}{\labelwidth}%
      \setlength{\labelsep}{\biblabelsep}%
      \addtolength{\leftmargin}{\labelsep}%
      \addtolength{\leftmargin}{\cvindent}%<----- here
      \setlength{\itemsep}{\bibitemsep}%
      \setlength{\parsep}{\bibparsep}}%
      \renewcommand*{\makelabel}[1]{\hss##1}}
  {\endlist}
  {\item}

% DOCUMENT LAYOUT
\usepackage{geometry}
\geometry{left=0.6in,right=0.6in,bottom=0.6in,top=0.6in,headsep=0.6cm, footskip=.6cm,letterpaper}
\setlength\parindent{0in}

% Underlining:
\usepackage[normalem]{ulem}

\usepackage{tagging}
\usepackage{calc}

% Setup section headers:
% using sectsy
% \usepackage{sectsty}
% \sectionfont{\mdseries\upshape\Large}
% \subsectionfont{\mdseries\scshape\normalsize}
% \subsubsectionfont{\mdseries\upshape\large}
\usepackage[explicit]{titlesec}
\titleformat{name=\section,numberless}{\normalfont\upshape\Large\bfseries}{}{0cm}{#1}[]
\titlespacing{\section}{0cm}{0.1cm}{0.1cm}
\titleformat{name=\subsection,numberless}{\normalfont\upshape\large\bfseries}{}{0.5cm}{#1}[]
\titlespacing{\subsection}{0cm}{0cm}{0.1cm}
% \titleformat{\section}
% {\normalfont\Large\bfseries}{\thesection}{1em}{}[after]
% \titleformat{\subsection}
% {\normalfont\large\bfseries}{\thesubsection}{1em}{}

%% Here I set up the headings:
\usepackage{fancyhdr}
\usepackage{datetime}
\pagestyle{fancy}
\rhead{}
\chead{\ifthenelse{\value{page}=1}{}{Michael Sobrepera - CV}}
\lhead{}
\renewcommand{\headrulewidth}{0pt}
\renewcommand{\footrulewidth}{1pt}
\rfoot{Compiled on \today\ at \currenttime ~EDT}
\cfoot{}
\lfoot{Page \thepage}

% ---- CUSTOM COMMANDS
\newcommand{\html}[1]{\href{#1}{\scriptsize\textsc{[html]}}}
\newcommand{\pdf}[1]{\href{#1}{\scriptsize\textsc{[pdf]}}}
\newcommand{\doi}[1]{\href{#1}{\scriptsize\textsc{[doi]}}}
% % ---- MARGIN YEARS
% \usepackage{marginnote}
% \newcommand{\years}[1]{\marginnote{\scriptsize #1}}
% \renewcommand*{\raggedleftmarginnote}{}
% \setlength{\marginparsep}{7pt}
% \reversemarginpar

% PDF SETUP
% ---- FILL IN HERE THE DOC TITLE AND AUTHOR
\usepackage{xurl}
\usepackage[bookmarks, colorlinks, breaklinks,
% ---- FILL IN HERE THE TITLE AND AUTHOR
    pdftitle={MichaelSobrepera-CV},
    pdfauthor={Michael Sobrepera},
    pdfsubject={curriculum vitae},
    pdfproducer={https://github.com/mjsobrep/cv/blob/master/MichaelSobrepera-CV.tex}
]{hyperref}
\hypersetup{linkcolor=blue,citecolor=blue,filecolor=black,urlcolor=urlblue}

% Define a new environment for sections in the CV.
% This environment takes 1 arg, the name of the section
\newenvironment{cvsection}[1]{
    % begining of environment:
    \section*{#1}
}{
    % end of environment:
}

% Define a new environment for sub sections in the CV.
% This environment takes 1 arg, the name of the section
\newenvironment{cvsubsection}[1]{
    % begining of environment:
    \subsection*{#1}
}{
    % end of environment:
}

% Define a new environment for items in the cv
% this takes args:
% 	1 - Title
%	2 - Company/group/univ
%	3 - Location
%	4 - start date
%	5 - end date
%	6 - description (optional)
%
% Ex: \cvitem{title}{company}{location}{start date}{end date}{description}
\newcommand{\cvitem}[6]{
    \setlength{\currentparskip}{\parskip}% save the value
    \strut\hfill\begin{minipage}{\dimexpr\textwidth-\cvindent}
    \setlength{\parskip}{\currentparskip}% restore the value
    \begin{tabular*}{\linewidth}{@{}l@{\extracolsep{\fill}}r@{}}
        \textbf{#1} & #4 -- #5\\
        \emph{#2} & \tagged{location}{#3}
    \end{tabular*}
    \ifx&#6&%
    \else
        \\[.05cm]
        #6
    \fi
    \end{minipage}
    %\vspace{.3cm}
}

\newcommand{\cvitemshort}[3]{
    \setlength{\currentparskip}{\parskip}% save the value
    \strut\hfill\begin{minipage}{\dimexpr\textwidth-\cvindent}
    \setlength{\parskip}{\currentparskip}% restore the value
    \begin{tabular*}{\linewidth}{@{}l@{\extracolsep{\fill}}r@{}}
        \textbf{#1} & #2
    \setbox0=\hbox{#3\unskip}\ifdim\wd0=0pt
        % third arg is empty
        \end{tabular*}
    \end{minipage}
    \vspace{-.2cm}
    \else
        % third arg is not empty
        \end{tabular*}\\[.05cm]
        #3
    \end{minipage}
    \fi
}

% \cvmentee{name}{start date}{end date}{description}
\newcommand{\cvmentee}[6]{
    \setlength{\currentparskip}{\parskip}% save the value
    \strut\hfill\begin{minipage}{\dimexpr\textwidth-\cvindent}
    \setlength{\parskip}{\currentparskip}% restore the value
    \begin{tabular*}{\linewidth}{@{}l@{\extracolsep{\fill}}r@{}}
        \textbf{#1} & #2 -- #3\\
        #4 & % #5
    \end{tabular*}\\[.05cm]
    #6
    \end{minipage}
    \vspace{.3cm}
}


%\usetag{location}
\usetag{short-mentees}
%\usetag{mentees}
\usetag{conference-abstracts}
\usetag{academia}
\usetag{industrial}

% DOCUMENT
\begin{document}

%%%%%%%%%%%% HEADER %%%%%%%%%%%%
\begin{center}
    {\LARGE Michael Sobrepera}\\[.5cm]
    University of Pennsylvania\\
    \begin{minipage}[c]{.35\textwidth}
    220 South 33rd Street\\
    229 Towne Building\\
    Philadelphia, PA 19104\\[.2cm]
    \end{minipage}
    \hspace{.28\textwidth}
    \begin{minipage}[r]{.35\textwidth}
    \raggedleft
    770--324--6196\\
    \href{mailto:mjsobrep@live.com}{mjsobrep@live.com}\\
    \href{mailto:mjsobrep@seas.upenn.edu}{mjsobrep@seas.upenn.edu}\\
    \href{http://michaelsobrepera.com}{michaelsobrepera.com}\\
    \end{minipage}
    \end{center}
    %%%%%%%%%%%%%%%%%%%%%%%%%%%%%%%
    \hrule
    \vspace{.4cm}


\section*{Summary}
\hspace*{\fill}\begin{minipage}{\textwidth-\cvindent}
I am a doctoral student and full stack roboticist.
I have an array of hard skills across design, hardware, software, and algorithms, complemented by strong data analysis, experimental design, and communication skills, allowing me to enter new project spaces, rapidly gain domain knowledge, and go deep into problems to generate answers and solutions. 
%My work focuses on understanding how social robots can be used to enhance telerehabilitation and how computer vision can be used for motor assessment. 
%I am funded under an NIH F31 Pre-Doctoral Fellowship. 
I am passionate about learning, working on meaningful problems, pushing the boundaries of technology, and mentoring the next generation.  

My PhD work has focused on understanding the use of social robotics for upper extremity rehabilitation and computer vision for objective assessment in upper extremity rehabilitation.
I have prior experience in computer vision for industrial automation and medical device design and manufacturing. 

\textbf{I am currently searching for a team working on exciting problems to join for a summer 2021 internship in Seattle.}
\end{minipage}

\begin{cvsection}{Education}

    \cvitem{Doctor of Philosophy in Mechanical Engineering}{University of Pennsylvania}{Philadelphia, PA}{Aug 2016}{Expected Dec 2021}{Advisor: Dr.\ Michelle Johnson. \\Affiliated with the General Robotics, Automation, Sensing, \& Perception Laboratory (GRASP Lab).}

    \cvitem{Master of Science in Robotics}{University of Pennsylvania}{Philadelphia, PA}{Aug 2016}{Aug 2019}{}
    %GPA: 3.32

    \cvitem{Bachelor of Science in Biomedical Engineering}{Georgia Institute of Technology}{Atlanta, GA}{Aug 2012}{Dec 2015}{Minor in Computer Science}
 %GPA: 3.35

    %\cvitem{Bachelor of Science in Chemical Engineering}{Auburn University Honors College}{Auburn, AL}{Aug 2011}{May 2012}{Pursued degree.}
 %GPA: 3.90
\end{cvsection}

\begin{cvsection}{Skills}

    \cvitemshort{Programming Languages}{Python, R, MATLAB, TypeScript, SQL, C++}{}

    \cvitemshort{Robotics}{ROS, HRI, Kinematics, State Estimation, System Integration, Mechatronics}{}

    \cvitemshort{Mechanical Design}{MCAD, Technical Drawing, RTV Molding, 3D Printing, Classical Machining}{}

    \cvitemshort{Software Infrastructure}{PyTorch, Docker, Git, NGINX, NodeJS, Redis, PostgreSQL, React}{}

\end{cvsection}
\begin{cvsection}{Funding}
     
    \cvitemshort{NIH F31 Predoctoral Fellowship (F31HD102165)}{Apr 2020 -- Present}{}

    \cvitemshort{University of Pennsylvania Fontaine Fellowship}{Sept 2016 -- Apr 2020}{}

\end{cvsection}

\begin{cvsection}{Experience}

    \cvitem{PhD Student}{University of Pennsylvania, Rehabilitation Robotics Laboratory}{Philadelphia, PA }{Aug 2016}{present}{
        Socially Assistive Robot for Upper Extremity Telerehabilitation 
        \begin{itemize}
            \item Led hardware design and software integration for the development of an affordable socially assistive robot (Lil'Flo) to aid in telepresence based assessment and treatment of patients with upper extremity motor impairments (\url{https://youtu.be/DDZe1RhcpWY}).
            \item Developed and now implementing experiments to determine how patients react to telepresence robots which incorporate social robots and how that affects remote assessment. 
            \item Developing a framework for identifying motor function from video of a patient doing various robot guided activities using both classical computer vision and machine learning techniques.
            \item Mentored and managed over a dozen students doing research within the project.
            \item Presented work in papers, posters, and talks. 
            \end{itemize}
}

    \cvitemtwojob{Research Technician II}{Georgia Institute of Technology, IRIM Technology Transition Laboratory}{Atlanta, GA}{Dec 2015}{Jun 2016}{
   Edge-Based Tracking for Flexible Manufacturing 
   \begin{itemize}
       \item Refined a C++ video based real-time textureless tracker from a research code base to a well documented robust system capable of running at \(30+\) fps at \(1920\times1080\) pixels, to enable manipulation on non-fixed, non-located car parts at a partner automotive facility. 
       \item Supported industrial partner in successful technical demonstrations to management. 
       %\item Utilized heavy industrial robots while developing perception, calibration, and control systems.
       \item Developed tools for calibrating multiple robot arms to cameras. 
       \item Integrated perception and motion control to track moving targets with a collaborative robot. 
   \end{itemize}
   }{Undergraduate Research Assistant}{Aug 2015}{Dec 2015}


    \cvitem{Automation Intern}{Eli Lilly and Company}{Indianapolis, IN}{May 2015}{Aug 2015}{
        Offline Plant Simulations for Automation Development and Testing
        \begin{itemize}
            \item Rapidly learned automation systems being used (Emerson DeltaV and Rockwell).
            %\item Identified, learned, and tested software packages for simulations.
            \item Evaluated options for offline software/hardware/operator in the loop plant simulations for process validation, control code development, pre-factory acceptance testing control system checkout, operator training, and process improvement.
            \item Reported on findings in both a technical paper and oral presentation.
\end{itemize}}

    \cvitem{Machine Shop Supervisor}{Georgia Institute of Technology, TEP Machine Shop}{Atlanta, GA}{Aug 2014}{May 2015}{
        \vspace{-0.2cm}
    \begin{itemize}
    \item Guided Master's in Biomedical Innovation and Development students in design and prototyping of medical devices.
    \item Supported the Cardiovascular Fluid Mechanics Lab and the Tissue Mechanics Lab in development and fabrication of experimental equipment. 
\end{itemize}
}

    \cvitem{Product Development Engineering Co-Op}{Unilife Corporation}{York/King of Prussia, PA}{Jan 2014}{Jul 2014}{
    % Developed procedure for rapidly testing design iterations using a hand powered injection molding machine.
    % Analyzed and modified product components and assemblies to improve final product outcomes, including problem assessment, design changes, prototype production, prototype testing, and submittal for final design review.
    % Designed, installed, and programmed automation tooling for the testing, manufacturing, and finishing of injectable drug delivery devices, including applications in micro injection molding, gluing, and small scale vacuum stoppering.
    % Independently developed and sourced custom product packaging materials from a foreign supplier.
    % Independently developed and sourced custom thermoformed components from a domestic supplier.
    % Sourced majority of custom components for pilot production (30 ppm) line, including BOM generation, drawing organization, RFQs, quote selection, and vendor relations.
    % Designed and sourced safety/airflow enclosures for pilot production line.
    Product Development and Manufacturing for Injectable Drug Delivery Devices
    \begin{itemize}
        \item Tested prototypes for both usability and engineering constraints and iterated on design. 
        \item Developed and prototyped new product concepts based on customer needs.
        \item Designed, procured, assembled, and programmed automation equipment for syringe component gluing, assembly, and finish operations. 
        \item Worked with vendor to design and procure sterilizable packaging for 1MM annual units of product.
    \end{itemize}
    %Worked throughout the product development pipeline on injectable drug delivery devices. Took products from customer needs through ideation, iterative prototyping, and pilot production. Worked on molding and assembly line design, procurement, assembly, and qualification. Developed test fixtures to ensure product quality.

    }

    % \cvitem{Custodian}{Cobb County School District, Kennesaw Mountain High School}{Kennesaw, GA}{Jun 2012}{Aug 2012}{Cleaned floors, tables, desks, bathrooms, etc., removed trash, stripped floors, waxed floors.}

    % \cvitem{Independent Contractor, Entertainment Specialist}{Fun-Fare}{Metro Atlanta, Ga}{April 2010}{Jun 2012}{
    % Worked setting up/dismantling inflatable rides, selling tickets, collecting tickets and supervising/assisting children on rides. Provided an enjoyable experience to children on the rides. Portrayed a positive image of the company, while acting as the point of sale for all tickets at events.
    % }

    % \cvitem{Histotechnician, Intern }{Emory-Adventist Hospital Anatomic Pathology Lab}{Smyrna, GA }{Aug 2010}{Dec 2010}{Processed specimens (grossing, embedding, cutting, and staining), researched H. pylori (bacteria).}

    % \cvitem{Medical Assistant, Administrative Assistant}{Metairie Diabetes Metabolic Clinic}{Metairie/LaPlace, LA}{May 2010}{Jul 2010}{Triaged (took vitals, drew blood, checked patients’ medications), coded for insurance billing, collected payment, answered phones.}

\end{cvsection}

\begin{cvsection}{Publications}


    \newrefsection
    \nocite{sobrepera2021DesignLilFlo}
    \nocite{sobrepera2021PerceivedUsefulnessSocial}
    \nocite{johnsonDesignAffordableSocially2019}
    \printbibliography[title={Peer-reviewed Journal Publications},heading=subbibliography]

    \newrefsection
    \nocite{sobreperaDesigningEvaluatingFace2019}
    \printbibliography[title={Peer-reviewed Conference Publications}, heading=subbibliography]

    %\newrefsection
    %\printbibliography[title={Preprints}, heading=subbibliography]

    \begin{taggedblock}{conference-abstracts}
    \newrefsection
    \nocite{sobrepera2020DesignLilFlo2,
        sobrepera2019DesigningArmsLil,
        tamakloe2019DesigningGame,
        resnageneral,
        resnaface
    }
    \printbibliography[title={Extended Conference Abstracts with Poster Presentations},heading=subbibliography]
\end{taggedblock}

\end{cvsection}

\begin{cvsection}{Awards \& Honors}

    \cvitemshort{Penn Wharton Entrepreneurship Startup Challenge Innovation Award}{May 2020}{}

    \cvitemshort{Rothberg Catalyzer First Place}{Nov 2019}{}


    \cvitemshort{Hispanic Scholarship Fund (HSF) Scholar}{Dec 2016}{}

    \cvitemshort{Georgia Institute of Technology OMED Tower Award}{2015}{}

    \cvitemshort{Georgia Institute of Technology Dean's List}{2012 -- 2014}{}

    \cvitemshort{Auburn University Dean's List}{2011 -- 2012}{}

    \cvitemshort{The Auburn National Scholars Presidential Scholarship}{2011}{}

\end{cvsection}


\begin{cvsection}{Ventures}

    \cvitem{MAR Orthotics}{Co-Founder \& President}{}{Oct 2019}{Feb 2021}{
        Novel Orthoses for Pediatric Cerebral Palsy 
        \begin{itemize}
            \item Public face of company, successfully pitched through multiple innovation and business competitions to win competitive awards.
            \item Performing customer discovery and validation to refine product market fit. 
            \item Working on technical design. 
        \end{itemize}
    }

\end{cvsection}

\begin{cvsection}{Professional Development}
    \cvitemshort{Neuromatch Academy}{Summer 2020}{}
\end{cvsection}

\begin{cvsection}{Teaching Experience}


\begin{cvsubsection}{Teaching Assistantships}

        \cvitemshort{Lead Teaching Assistant for MEAM 147: Intro to Mechanics Lab}{Fall 2018}{}
        %{University of Pennsylvania}{Philadelphia, PA}{Aug 2018}{Dec 2018}{Organized and ran the mechanics lab class for first semester undergraduate students in Mechanical Engineering and related fields, managing a teaching team of 6. The class consisted of 41 students across 3 sections.}

        \cvitemshort{Teaching Assistant for MEAM 211: Undergraduate Dynamics}{Spring 2018}{}
        %{Designed and ran problem based recitation sections, held office hours,  covered 2 general lectures, developed a problem set, graded exams, and managed 5 graders. The class consisted of 56 students}

        \cvitemshort{Teaching Assistant for MEAM 147: Intro to Mechanics Lab}{Fall 2017}{}
        %{Assisted in the management of the mechanics lab class for first semester undergraduate students in Mechanical Engineering and related fields. The class consisted of 36 students across 3 sections}

    \end{cvsubsection}

    \begin{cvsubsection}{Guest Lectures}

        \cvitemshort{Robots in Pediatric Rehabilitation}{Nov 2018, 2020}{Course: Bioengineering 514: Rehab Engineering and Design}

       % Lecture on Robotics in Pediatric Rehabilitation given to the \textit{Bioengineering 514: Rehab Engineering and Design} class at the University of Pennsylvania, including demos of systems, and a final group activity.}

        \cvitemshort{Robots in Pediatric Rehabilitation}{Apr 2018}{Course: Robots in HealthCare: From Science Fiction to Reality}

        %Lecture on Robotics in Pediatric Rehabilitation given to the Teaching Institute of Philadelphia \textit{Robots in HealthCare: From Science Fiction to Reality} class.}

        \cvitemshort{Robot Inspiration: Artificial Intelligence, the Brain, and Programming}{Mar 2018}{Course: Robots in HealthCare: From Science Fiction to Reality}

        %Lecture on the basics of artificial intelligence and machine learning given to the Teaching Institute of Philadelphia \textit{Robots in HealthCare: From Science Fiction to Reality} class.}

    \end{cvsubsection}

\end{cvsection}

\begin{cvsection}{Talks}
    \cvitemshort{Global Perspectives on Medicine, Rehabilitation, and Robotics Webinar Series}{Nov 2020}{Using Social Robots for Remote Assessment of Children With Disability}

    \cvitemshort{Penn MEAM Department Seminar}{Jul 2020}{The Design of Lil'Flo, a Socially Assistive Robot for Upper Extremity Motor Assessment and Rehabilitation Via Telepresence}
\end{cvsection}


\begin{cvsection}{Service}
\begin{cvsubsection}{Community Service}

    \cvitemshort{UPenn Bioengineering BETA Day Volunteer}{Jan 2020}{}

    \cvitemshort{Tech Girlz Circuit Workshop Volunteer}{Jan 2020}{}

    \cvitemshort{Girls Advancing in STEM (GAINS) Lab Tour Lead}{Nov 2019}{}

    \cvitemshort{Philadelphia Robotics Expo Volunteer}{Oct 2019}{}

    \cvitemshort{Philadelphia Maker Faire Volunteer}{Oct 2019}{}

\cvitemshort{Upward Bound \textit{Growing out of the Stereotypes} Workshop Lead}{Dec 2018}{} 
%Developed and ran a workshop introducing high school students to programming using Sphero robots.} % Workshop was a  part of an event  by the Singh Center for Nanotechnology for the LULAC National Educational Service Centers (LNESC) Upward Bound Program.}

    \cvitemshort{EL Education Future of Work Conference Interviewee}{Nov 2018}{}

    \cvitemshort{Participated in RET: Leveraging Our Collective Impact Conference}{Oct 2018}{}

    \cvitemshort{Philadelphia Robotics Expo Presenter}{Oct 2018}{}

\cvitemshort{GRASP NSF Research Experience for Teachers Program}{Summer 2017 \& Summer 2018}{
    Mentored three middle school teachers through a six week research experience covering the entire scientific process. Visited their classrooms a total of six times to demonstrate state of the art research and teach lessons. 
    %Research Experience for Teachers
    %\begin{itemize}
    %    \item Mentored three School District of Philadelphia middle school teachers, two math and one science, through a six week research experience. Guided teachers through the process of establishing the state of the art in an area, identifying a knowledge gap, developing a hypothesis to narrow the gap, developing a methodology to test their hypothesis, testing their hypothesis, and presenting their work in written and oral form.
    %    \item Visited middle school classrooms six times to demonstrate work being done today in research  and help teach lessons.
    %    \item Delivered an introduction to circuits class and an introduction to computer aided design class to the cohort of teachers in the program.
    %\end{itemize}
    }

\cvitemshort{Be a Pennovator Workshop Lead}{Apr 2018}{}
%Developed and ran a workshop introducing middle school students to programming robots using Sphero robots.}

    \cvitemshort{Penn Science Olympiad Volunteer}{Feb 2017}{}

    \cvitemshort{Penn First Lego League Judge}{Feb 2017}{}

    \cvitemshort{Penn-Alexander School Science Fair Judge}{Dec 2016}{}

\end{cvsubsection}

\begin{cvsubsection}{University Service}

    \cvitemshort{Penn Doctoral Diversity and Inclusion Board Member}{Jun 2020 -- Present}{}

    %\cvitemshort{\makecell{Penn Engineering COVID-19 Research and Academic Safety\\Reporting Committee Member}}{Jun 2020 -- Present}{}

    \cvitemshort{GRASP Student Advisory Committee Member}{Jan 2020 -- Present}{}
    


    \cvitemshort{Mechanical Engineering Graduate Association Vice President}{Sept 2017 -- Sept 2018}{}
    %\cvitem{Mechanical Engineering Graduate Association Vice President}{University of Pennsylvania}{Philadelphia, PA}{Sept 2017}{Sept 2018}{Represented masters and PhD students in the department and organized events to foster academic cooperation and departmental unity. Started a  PhD mentoring program to ease the transition into the PhD by giving new students access to more experienced PhD students outside of their labs. Also started a  one hour lab showcase event to introduce  the research done across the department to incoming students through a series of 12 fast-paced talks.}



\end{cvsubsection}

\begin{cvsubsection}{Reviewer}

    \cvitemshort{Journal of Rehabilitation and Assistive Technologies Engineering}{2020}{}

    \cvitemshort{International Conference on Robotics and Automation}{2017, 2020}{}

    \cvitemshort{International Conference on Intelligent Robots and Systems}{2018, 2020}{}

    \cvitemshort{HRI Conference}{2020--2021}{}

\end{cvsubsection}

\begin{cvsubsection}{Conference Volunteer}
    
    \cvitemshort{Northeast Robotics Colloquium}{2019}{}

    \cvitemshort{Biomedical Engineering Society Conference}{2019}{}

    \cvitemshort{Rehabilitation Engineering and Assistive Technology Society of North America Conference}{2018}{}

\end{cvsubsection}

\end{cvsection}

\begin{cvsection}{Selected Press}

    %\cvitemshort{How Roboticists (and Robots) Have Been Working from Home}{Jun 2020}{\url{https://spectrum.ieee.org/automaton/robotics/home-robots/how-roboticists-and-robots-have-been-working-from-home}}

    \cvitemshort{2020 Startup Challenge Special Part 3: MAR Designs}{Apr 2020}{\url{https://link.medium.com/34ti0GvN83}}

    \cvitemshort{Students' Innovative Orthotic Device Wins Rothberg Catalyzer}{Oct 2019}{\url{https://link.medium.com/34ti0GvN83}}

    \cvitemshort{\shortstack[l]{Teachers Become Students to Become Better Teachers \\at GRASP Lab’s RET Program}}{Sept 2018}{\url{https://link.medium.com/zIHvqZH25S}}

\end{cvsection}

\begin{taggedblock}{mentees,short-mentees}
\begin{cvsection}{Mentees}

    \cvmentee{Vera Lee}{Sept 2019}{Present}{Penn BioE Undergrad, Robotics Master's}{Senior Fall}{Working to drive forward the Flo project, by working on mechanical design and supporting subject testing.}

    \cvmentee{Suveer Garg}{Feb 2020}{Present}{Penn Systems Engineering Master's}{First year Master's}{Working on the perception system for the Flo robots. Developing algorithms for the analysis of upper extremity function using computer vision.}

    \cvmentee{Ralph Tamakloe}{Jun 2019}{Aug 2019}{Penn BioE Undergrad}{Freshman Summer}{Worked to understand how desktop social robots can be used for low cost, portable, objective assesments.}

    \cvmentee{Dhruv Karthik}{Jan 2018}{May 2019}{Penn CIS Undergrad}{Freshman Fall}{Worked to build automatically generated syntactic maps of hospitals to lower the barriers to entry of mobile robots in hospitals.}

    \cvmentee{Enri Kina}{May 2017}{May 2019}{Penn MEAM Undergrad}{Freshman Summer -- Junior Spring}{Worked heavily on designing the head for a social robot and understanding how to test the designs. Work was done as part of the Penn Rachleff Scholars Program.}

    \cvmentee{Danielle Chen}{Jun 2018}{Sept 2018}{Penn Integrated Product Design Master's}{First Year Summer}{Worked on the form and design of a social assistive robot.}

    \cvmentee{Andrew Levine}{Jun 2018}{Aug 2018}{Penn MEAM Undergrad}{Sophomore Summer}{Worked to program a Nao robot to perform neuromotor therapy exercises with patients.}

    \cvmentee{Jagtar Singh}{May 2018}{Aug 2018}{Penn MEAM Master's}{First Year Summer}{Worked towards a framework to extract biometric markers, such as pulse rate from a video of an active person.}

    \cvmentee{Shyon Small}{May 2018}{Jul 2018}{Penn BioE Undergrad}{Sophomore Summer}{Explored the use of social robots to perform cognitive testing on pediatric patients with motor disorders such as Cerebral Palsy. Completed  work as part of the National Science Foundation and Philadelphia Region Louis Stokes Alliance for Minority Participation (NSF/LSAMP) Undergraduate Research Program.}

    \cvmentee{Weiyu Du}{May 2018}{Jul 2018}{Penn CIS Undergrad}{Freshman Summer}{Developed an understanding of challenges and opportunities presented by hospital environments for low cost autonomous navigation. Worked as part of the Penn Undergraduate Research Mentoring Program (PURM).}

    \cvmentee{Sabrina Smith}{July 2017}{Sept 2017}{Imperial College London Biomedical Engineering Undergrad}{idk}{Worked on algorithms to convert outputs from human motion tracking systems to features which provide diagnostic information on upper extremity impairment.}

    \cvmentee{Tim Kulesza}{Jun 2017}{Aug 2017}{BSE, Mechanical Engineering \& Materials Science}{Summer After Graduation}{Developed a set of needs for a mobile robotic base to carry a hospital based semi autonomous social robot. Translated the needs to a series of design requirements and developed specifications for such a system.}

    \cvmentee{Leora Korn}{May 2017}{July 2017}{Penn MEAM Undergrad}{Freshman Summer}{Worked on designing the upper extremities of a social assistive robot for upper extremity rehab so  they could adequately perform tasks necessary to engage patients in rehab activities.}

\end{cvsection}
\end{taggedblock}

\end{document}

\documentclass[10pt, letter]{article}

\usepackage[utf8]{inputenc}
\usepackage[english]{babel}
\usepackage{microtype}

\usepackage{xcolor}
\definecolor{urlblue}{HTML}{01256e}
\usepackage[inline]{enumitem}

\usepackage{lipsum}% http://ctan.org/pkg/lipsum

\usepackage{booktabs}
\usepackage{array}

% Spacing between paragraphs
\usepackage{parskip}
\newlength{\currentparskip}

%%%%% Indentation of content
\newlength{\cvindent}
\setlength{\cvindent}{1cm}

% bibliography setup
\usepackage[maxbibnames=99,sorting=none]{biblatex}
\addbibresource{ref.bib}
\defbibenvironment{bibliography}
  {\list
     {\printtext[labelnumberwidth]{%
      \printfield{labelprefix}%
      \printfield{labelnumber}}}
     {\setlength{\labelwidth}{\labelnumberwidth}%
      \setlength{\leftmargin}{\labelwidth}%
      \setlength{\labelsep}{\biblabelsep}%
      \addtolength{\leftmargin}{\labelsep}%
      \addtolength{\leftmargin}{\cvindent}%<----- here
      \setlength{\itemsep}{\bibitemsep}%
      \setlength{\parsep}{\bibparsep}}%
      \renewcommand*{\makelabel}[1]{\hss##1}}
  {\endlist}
  {\item}

% DOCUMENT LAYOUT
\usepackage{geometry} 
\geometry{margin=1in,letterpaper}
\setlength\parindent{0in}

% Underlining:
\usepackage[normalem]{ulem} 

% Setup section headers:
% using sectsy
% \usepackage{sectsty} 
% \sectionfont{\mdseries\upshape\Large}
% \subsectionfont{\mdseries\scshape\normalsize} 
% \subsubsectionfont{\mdseries\upshape\large} 
\usepackage[explicit]{titlesec}
\titleformat{name=\section,numberless}{\normalfont\upshape\Large\bfseries}{}{0em}{#1}[]
\titleformat{name=\subsection,numberless}{\normalfont\upshape\large\bfseries}{}{0.5cm}{#1}[]
% \titleformat{\section}
% {\normalfont\Large\bfseries}{\thesection}{1em}{}[after]
% \titleformat{\subsection}
% {\normalfont\large\bfseries}{\thesubsection}{1em}{}

%% Here I set up the headings:
\usepackage{fancyhdr}
\usepackage{datetime}
\pagestyle{fancy}
\rhead{}
\chead{\ifthenelse{\value{page}=1}{}{Michael Sobrepera}}
\lhead{}
\renewcommand{\headrulewidth}{0pt}
\renewcommand{\footrulewidth}{1pt}
\rfoot{Compiled on \today\ at \currenttime  ~EDT}
\cfoot{}
\lfoot{Page \thepage}

% ---- CUSTOM COMMANDS
\newcommand{\html}[1]{\href{#1}{\scriptsize\textsc{[html]}}}
\newcommand{\pdf}[1]{\href{#1}{\scriptsize\textsc{[pdf]}}}
\newcommand{\doi}[1]{\href{#1}{\scriptsize\textsc{[doi]}}}
% % ---- MARGIN YEARS
% \usepackage{marginnote}
% \newcommand{\years}[1]{\marginnote{\scriptsize #1}}
% \renewcommand*{\raggedleftmarginnote}{}
% \setlength{\marginparsep}{7pt}
% \reversemarginpar

% PDF SETUP
% ---- FILL IN HERE THE DOC TITLE AND AUTHOR
\usepackage[bookmarks, colorlinks, breaklinks, 
% ---- FILL IN HERE THE TITLE AND AUTHOR
    pdftitle={MichaelSobrepera-CV},
    pdfauthor={Michael Sobrepera},
    pdfsubject={curriculum vitae},
    pdfproducer={https://github.com/mjsobrep/cv/blob/master/MichaelSobrepera-CV.tex}
]{hyperref}  
\hypersetup{linkcolor=blue,citecolor=blue,filecolor=black,urlcolor=urlblue} 


% Define a new environment for sections in the CV.
% This environment takes 1 arg, the name of the section
\newenvironment{cvsection}[1]{
    % begining of environment:
    \section*{#1}   
}{
    % end of environment:
}

% Define a new environment for sub sections in the CV.
% This environment takes 1 arg, the name of the section
\newenvironment{cvsubsection}[1]{
    % begining of environment:
    \subsection*{#1}
}{
    % end of environment:
}

% Define a new environment for items in the cv
% this takes args: 
% 	1 - Title
%	2 - Company/group/univ
%	3 - Location
%	4 - start date
%	5 - end date
%	6 - description (optional)
%
% Ex: \cvitem{title}{company}{location}{start date}{end date}{description}
\newcommand{\cvitem}[6]{
    \setlength{\currentparskip}{\parskip}% save the value
    \strut\hfill\begin{minipage}{\dimexpr\textwidth-\cvindent}
    \setlength{\parskip}{\currentparskip}% restore the value
    \begin{tabular*}{\linewidth}{@{}l@{\extracolsep{\fill}}r@{}}
        \textbf{#1} & #4 -- #5\\
        #2 & #3
    \end{tabular*}\\[.05cm]
    #6
    \end{minipage}
    \vspace{.3cm}
}

\newcommand{\cvitemshort}[3]{
    \setlength{\currentparskip}{\parskip}% save the value
    \strut\hfill\begin{minipage}{\dimexpr\textwidth-\cvindent}
    \setlength{\parskip}{\currentparskip}% restore the value
    \begin{tabular*}{\linewidth}{@{}l@{\extracolsep{\fill}}r@{}}
        \textbf{#1} & #2
    \setbox0=\hbox{#3\unskip}\ifdim\wd0=0pt
        % third arg is empty
        \end{tabular*}
    \else
        % third arg is not empty
        \end{tabular*}\\[.05cm]
        #3
    \fi
    \end{minipage}
    \vspace{.3cm}
}

% \cvmentee{name}{start date}{end date}{description}
\newcommand{\cvmentee}[4]{
    \setlength{\currentparskip}{\parskip}% save the value
    \strut\hfill\begin{minipage}{\dimexpr\textwidth-\cvindent}
    \setlength{\parskip}{\currentparskip}% restore the value
    \begin{tabular*}{\linewidth}{@{}l@{\extracolsep{\fill}}r@{}}
        \textbf{#1} & #2 -- #3
    \end{tabular*}\\[.05cm]
    #4
    \end{minipage}
    \vspace{.3cm}
}


% DOCUMENT
\begin{document}

%%%%%%%%%%%% HEADER %%%%%%%%%%%%
\begin{center}
    {\LARGE Michael Sobrepera}\\[.5cm]
    University of Pennsylvania\\
    \begin{minipage}[c]{.35\textwidth}
    220 South 33rd Street\\
    229 Towne Building\\
    Philadelphia, PA 19104\\[.2cm]
    \end{minipage}
    \hspace{.28\textwidth}
    \begin{minipage}[r]{.35\textwidth}
    \raggedleft
    770--324--6196\\
    \href{mailto:mjsobrep@live.com}{mjsobrep@live.com}\\
    \href{mailto:mjsobrep@seas.upenn.edu}{mjsobrep@seas.upenn.edu}\\
    \href{http://michaelsobrepera.com}{michaelsobrepera.com}\\
    \end{minipage}
    \end{center}
    %%%%%%%%%%%%%%%%%%%%%%%%%%%%%%%
    \hrule
    \vspace{.4cm}


\section*{Summary}
PhD student and full stack roboticist with background in medical device design and manufacturing. Working to apply social robots to upper extremity rehabilitation.

\begin{cvsection}{Education}
    \cvitem{Doctor of Philosophy in Mechanical Engineering}{University of Pennsylvania}{Philadelphia, PA}{Aug 2016}{Expected Apr 2021}{PhD student in Mechanical Engineering and Applied Mechanics (MEAM). Adviser: Dr.\ Michelle Johnson, The Rehabilitation Robotics Laboratory. Affiliated with the General Robotics, Automation, Sensing, \& Perception Laboratory (GRASP lab).}
 %GPA: 3.28

    \cvitem{Bachelor of Science in Biomedical Engineering}{Georgia Institute of Technology}{Atlanta, GA}{Aug 2012}{Dec 2015}{Minor in Computer Science}
 %GPA: 3.35

    %\cvitem{Bachelor of Science in Chemical Engineering}{Auburn University Honors College}{Auburn, AL}{Aug 2011}{May 2012}{Pursued degree.}
 %GPA: 3.90
\end{cvsection}

\begin{cvsection}{Research Experience}
    \cvitem{Socially Assistive Robot for Upper Extremity Telerehab}{University of Pennsylvania, Rehabilitation Robotics Laboratory}{Philadelphia, PA }{Aug 2016}{present}{
    Developing an affordable socially assistive robot to aid in telepresence based assessment and treatment of patients with upper extremity motor impairments as a tool to expand the reach of clinicians while improving the objectivity of diagnostic tools.
The project involves two primary foci 1) understanding how patients react to telepresence robots which incorporate social robots and how that affects remote assessment and 2) developing a framework for identifying motor function from video of a patient doing various robot guided activities.}

    \cvitem{Edge-Based Tracking for Flexible Manufacturing}{Georgia Institute of Technology, IRIM Technology Transition Laboratory}{Atlanta, GA}{Aug 2015}{June 2016}{Worked as an Undergraduate Researcher (Aug 2015 -- Dec 2015) and Research Technician II (Dec 2015 -- June 2016) under Dr.\ Henrik Christensen and Dr.\ Larry Sweet to advance technologies in the visual servoing space, explicitly focused on potential industry applications. Refined an existing video based real-time textureless tracker from a research code base to a well documented robust system capable of running at \(30+\) fps at \(1920\times1080\) pixels, to enable manipulation on non-fixed, non-located car parts at a client automotive facility, resulting in successful demonstrations to management. Utilized heavy industrial robots while developing perception, calibration, and control systems in C++ and Python. }
\end{cvsection}

\begin{cvsection}{Publications}
    \newrefsection
    \nocite{johnsonDesignAffordableSocially2019}
    \printbibliography[title={Journal Papers},heading=subbibliography]

    \newrefsection
    \nocite{sobreperaDesigningEvaluatingFace2019}
    \printbibliography[title={Conference Papers}, heading=subbibliography]

    \newrefsection
    \nocite{sobrepera2019DesigningArmsLil,resnageneral,resnaface}
    \printbibliography[title={Conference Abstracts with Poster Presentations},heading=subbibliography]
\end{cvsection}

\begin{cvsection}{Teaching}
    \begin{cvsubsection}{Teaching Assistantships}
        \cvitem{MEAM 147: Intro to Mechanics Lab}{University of Pennsylvania}{Philadelphia, PA}{Aug 2018}{Dec 2018}{Organized and ran the mechanics lab class for first semester undergraduate students in Mechanical Engineering and related fields, managing a teaching team of 6. The class consisted of 41 students across 3 sections.}

        \cvitem{MEAM 211: Undergraduate Dynamics}{University of Pennsylvania}{Philadelphia, PA}{Jan 2018}{May 2018}{Designed and ran problem based recitation sections, held office hours,  covered 2 general lectures, developed a problem set, graded exams, and managed 5 graders. The class consisted of 56 students}

        \cvitem{MEAM 147: Intro to Mechanics Lab}{University of Pennsylvania}{Philadelphia, PA}{Aug 2017}{Dec 2017}{Assisted in the management of the mechanics lab class for first semester undergraduate students in Mechanical Engineering and related fields. The class consisted of 36 students across 3 sections}
    \end{cvsubsection}

    \begin{cvsubsection}{Guest Lectures}
        \cvitemshort{Robots in Pediatric Rehabilitation}{Nov 2018}{Lecture on Robotics in Pediatric Rehabilitation given to the \textit{Bioengineering 514: Rehab Engineering and Design} class at the University of Pennsylvania, including demos of systems, and a final group activity.}

        \cvitemshort{Robots in Pediatric Rehabilitation}{Apr 2018}{Lecture on Robotics in Pediatric Rehabilitation given to the Teaching Institute of Philadelphia \textit{Robots in HealthCare: From Science Fiction to Reality} class.}

        \cvitemshort{Robot Inspiration: Artificial Intelligence/Brain and Programming}{Mar 2018}{Lecture on the basics of artificial intelligence and machine learning given to the Teaching Institute of Philadelphia \textit{Robots in HealthCare: From Science Fiction to Reality} class.}
    \end{cvsubsection}

    \begin{cvsubsection}{Mentees}
        \cvmentee{Vera Lee}{Sept 2019}{Present}{Penn BioE Undergrad}{Senior Fall}{Working to drive forward the Flo project, by working on mechanical design and supporting subject testing.}

        \cvmentee{Ralph Tamakloe}{Jun 2019}{Aug 2019}{Penn BioE Undergrad}{Freshman Summer}{Worked to understand how desktop social robots can be used for low cost, portable, objective assesments.}

        \cvmentee{Dhruv Karthik}{Jan 2018}{May 2019}{Penn CIS Undergrad}{Freshman Fall}{Worked to build automatically generated syntactic maps of hospitals to lower the barriers to entry of mobile robots in hospitals.}

        \cvmentee{Enri Kina}{May 2017}{May 2019}{Penn MEAM Undergrad}{Freshman Summer -- Junior Spring}{Worked heavily on designing the head for a social robot and understanding how to test the designs. Work was done as part of the Penn Rachleff Scholars Program.}

        \cvmentee{Danielle Chen}{Jun 2018}{Sept 2018}{Penn Integrated Product Design Masters}{First Year Summer}{Worked on the form and design of a social assistive robot.}

        \cvmentee{Andrew Levine}{Jun 2018}{Aug 2018}{Penn MEAM Undergrad}{Sophomore Summer}{Worked to program a Nao robot to perform neuromotor therapy exercises with patients.}

        \cvmentee{Jagtar Singh}{May 2018}{Aug 2018}{Penn MEAM Masters}{First Year Summer}{Worked towards a framework to extract biometric markers, such as pulse rate from a video of an active person.}

        \cvmentee{Shyon Small}{May 2018}{Jul 2018}{Penn BioE Undergrad}{Sophomore Summer}{Explored the use of social robots to perform cognitive testing on pediatric patients with motor disorders such as Cerebral Palsy. Completed  work as part of the National Science Foundation and Philadelphia Region Louis Stokes Alliance for Minority Participation (NSF/LSAMP) Undergraduate Research Program.}

        \cvmentee{Weiyu Du}{May 2018}{Jul 2018}{Penn CIS Undergrad}{Freshman Summer}{Developed an understanding of challenges and opportunities presented by hospital environments for low cost autonomous navigation. Worked as part of the Penn Undergraduate Research Mentoring Program (PURM).}

        \cvmentee{Sabrina Smith}{July 2017}{Sept 2017}{Imperial College London Biomedical Engineering Undergrad}{idk}{Worked on algorithms to convert outputs from human motion tracking systems to features which provide diagnostic information on upper extremity impairment.}

        \cvmentee{Tim Kulesza}{Jun 2017}{Aug 2017}{BSE, Mechanical Engineering \& Materials Science}{Summer After Graduation}{Developed a set of needs for a mobile robotic base to carry a hospital based semi autonomous social robot. Translated the needs to a series of design requirements and developed specifications for such a system.}

        \cvmentee{Leora Korn}{May 2017}{July 2017}{Penn MEAM Undergrad}{Freshman Summer}{Worked on designing the upper extremities of a social assistive robot for upper extremity rehab so  they could adequately perform tasks necessary to engage patients in rehab activities.}

    \end{cvsubsection}
\end{cvsection}

\begin{cvsection}{Service}
    \cvitemshort{UPenn Bioengineering BETA Day Volunteer}{Jan 2020}{}

    \cvitemshort{Tech Girlz Circuit Workshop Volunteer}{Jan 2020}{}

    \cvitemshort{Girls Advancing in STEM (GAINS) Lab Tour Lead}{Nov 2019}{}

    \cvitemshort{Philadelphia Robotics Expo Volunteer}{Oct 2019}{}

    \cvitemshort{Philadelphia Maker Faire Volunteer}{Oct 2019}{}

    \cvitemshort{Upward Bound \textit{Growing out of the Stereotypes} Workshop Lead}{Dec 2018}{Developed and ran a workshop introducing high school students to programming robots using Sphero robots and custom-designed wooden mazes. Workshop was a  part of an event  by the Singh Center for Nanotechnology for the LULAC National Educational Service Centers (LNESC) Upward Bound Program.}

    \cvitemshort{EL Education Future of Work Conference Interviewee}{Nov 2018}{}

    \cvitemshort{Participated in RET: Leveraging Our Collective Impact Conference}{Oct 2018}{}

    \cvitemshort{Philadelphia Robotics Expo Presenter}{Oct 2018}{}

    \cvitem{Mechanical Engineering Graduate Association Vice President}{University of Pennsylvania}{Philadelphia, PA}{Sept 2017}{Sept 2018}{Represented masters and PhD students in the department and organized events to foster academic cooperation and departmental unity. Started a  PhD mentoring program to ease the transition into the PhD by giving new students access to more experienced PhD students outside of their labs. Also started a  one hour lab showcase event to introduce  the research done across the department to incoming students through a series of 12 fast-paced talks.}

    \cvitem{GRASP RET Program}{University of Pennsylvania}{Philadelphia, PA}{Jun 2018}{Aug 2018}{Mentored two School District of Philadelphia middle school teachers, one math and one science, through a six week research experience. Guided teachers through the process of establishing the state of the art in an area, identifying a knowledge gap, developing a hypothesis to narrow the gap, developing a methodology to test their hypothesis, testing their hypothesis, and presenting their work in written and oral form.

    The program includes a total of six visits to middle school classrooms throughout the 2018 -- 2019 school year to demonstrate work being done today in research  and help teach lessons.

    Delivered an introduction to circuits class and an introduction to computer aided design class to the cohort of teachers in the program.

    Work funded by NSF Grant \#: 1542301}

    \cvitemshort{Be a Pennovator Workshop Lead}{Apr 2018}{Developed and ran a workshop introducing middle school students to programming robots using Sphero robots and custom-designed wooden mazes.}

    \cvitem{GRASP RET Program}{University of Pennsylvania}{Philadelphia, PA}{Jun 2017}{Aug 2017}{Mentored a School District of Philadelphia middle school science teacher through a six week research experience. Guided her through the process of establishing the state of the art in an area, identifying a knowledge gap, developing a hypothesis to narrow the gap, developing a methodology to test her hypothesis, testing her hypothesis, and presenting her work in written and oral form.

    Visited her classroom three times throughout the 2017-2018 school year to demonstrate work being done today in research, help teach a circuits lesson, and to lead a coding lesson using Spheros.

    Developed and delivered an introduction to circuits class and an introduction to computer aided design class to the cohort of teachers in the program.

    Work funded by NSF Grant \#: 1542301}

    \cvitemshort{Penn Science Olympiad Volunteer}{Feb 2017}{}

    \cvitemshort{Penn First Lego League Judge}{Feb 2017}{}

    \cvitemshort{Penn-Alexander School Science Fair Judge}{Dec 2016}{}
\end{cvsection}

\begin{cvsection}{Awards}
    \cvitemshort{Rothberg Catalyzer First Place}{Nov 2019}{}

    \cvitemshort{University of Pennsylvania Fontaine Fellowship}{2016 -- Present}{}

    \cvitemshort{Hispanic Scholarship Fund (HSF) Scholar}{Dec 2016}{}

    \cvitemshort{Georgia Institute of Technology OMED Tower Award}{2015}{}

    \cvitemshort{Georgia Institute of Technology Dean's List}{2012 -- 2014}{}

    \cvitemshort{Auburn University Dean's List}{2011 -- 2012}{}

\end{cvsection}

\begin{cvsection}{Work Experience}
    \cvitem{Automation Intern}{Eli Lilly and Company}{Indianapolis, IN}{May 2015}{Aug 2015}{Evaluated options for offline software/hardware/operator in the loop plant simulations for the purposes of process validation, control code development, pre-Factory Acceptance Testing control system checkout, operator training, and process improvement, with a specific focus on Emerson DeltaV and Rockwell interoperability. Reported on findings in both a technical paper and oral presentation.}

    \cvitem{Machine Shop Supervisor}{Georgia Institute of Technology, TEP Machine Shop}{Atlanta, GA}{Aug 2014}{May 2015}{Maintained equipment, trained shop users, guided Masters in Biomedical Innovation and Development (MBID) students in design and prototyping of medical devices, supported the Cardiovascular Fluid Mechanics (CFM) Lab and the Tissue Mechanics Lab. }

    \cvitem{Product Development Engineering CO-OP}{Unilife Corporation}{York/King of Prussia, PA}{Jan 2014}{Jul 2014}{
    % Developed procedure for rapidly testing design iterations using a hand powered injection molding machine.
    % Analyzed and modified product components and assemblies to improve final product outcomes, including problem assessment, design changes, prototype production, prototype testing, and submittal for final design review.
    % Designed, installed, and programmed automation tooling for the testing, manufacturing, and finishing of injectable drug delivery devices, including applications in micro injection molding, gluing, and small scale vacuum stoppering.
    % Independently developed and sourced custom product packaging materials from a foreign supplier.
    % Independently developed and sourced custom thermoformed components from a domestic supplier.
    % Sourced majority of custom components for pilot production (30 ppm) line, including BOM generation, drawing organization, RFQs, quote selection, and vendor relations.
    % Designed and sourced safety/airflow enclosures for pilot production line.
    Worked throughout the product development pipeline on injectable drug delivery devices. Took products from customer needs through ideation, iterative prototyping, and pilot production. Worked on molding and assembly line design, procurement, assembly, and qualification. Developed test fixtures to ensure product quality.
    }

    % \cvitem{Custodian}{Cobb County School District, Kennesaw Mountain High School}{Kennesaw, GA}{Jun 2012}{Aug 2012}{Cleaned floors, tables, desks, bathrooms, etc., removed trash, stripped floors, waxed floors.}

    % \cvitem{Independent Contractor, Entertainment Specialist}{Fun-Fare}{Metro Atlanta, Ga}{April 2010}{Jun 2012}{
    % Worked setting up/dismantling inflatable rides, selling tickets, collecting tickets and supervising/assisting children on rides. Provided an enjoyable experience to children on the rides. Portrayed a positive image of the company, while acting as the point of sale for all tickets at events.
    % }

    % \cvitem{Histotechnician, Intern }{Emory-Adventist Hospital Anatomic Pathology Lab}{Smyrna, GA }{Aug 2010}{Dec 2010}{Processed specimens (grossing, embedding, cutting, and staining), researched H. pylori (bacteria).}

    % \cvitem{Medical Assistant, Administrative Assistant}{Metairie Diabetes Metabolic Clinic}{Metairie/LaPlace, LA}{May 2010}{Jul 2010}{Triaged (took vitals, drew blood, checked patients’ medications), coded for insurance billing, collected payment, answered phones.}
\end{cvsection}

\begin{cvsection}{Press}
    \cvitemshort{Students' Innovative Orthotic Device Wins Rothberg Catalyzer}{Oct 2019}{\url{https://link.medium.com/34ti0GvN83}}

    \cvitemshort{\shortstack[l]{Teachers Become Students to Become Better Teachers \\at GRASP Lab’s RET Program}}{Sept 2018}{\url{https://link.medium.com/zIHvqZH25S}}
\end{cvsection}

\end{document}
